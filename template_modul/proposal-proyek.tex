\documentclass[11pt,]{article}
\usepackage{lmodern}
\usepackage{amssymb,amsmath}
\usepackage{ifxetex,ifluatex}
\usepackage{fixltx2e} % provides \textsubscript
\ifnum 0\ifxetex 1\fi\ifluatex 1\fi=0 % if pdftex
  \usepackage[T1]{fontenc}
  \usepackage[utf8]{inputenc}
\else % if luatex or xelatex
  \ifxetex
    \usepackage{mathspec}
  \else
    \usepackage{fontspec}
  \fi
  \defaultfontfeatures{Ligatures=TeX,Scale=MatchLowercase}
\fi
% use upquote if available, for straight quotes in verbatim environments
\IfFileExists{upquote.sty}{\usepackage{upquote}}{}
% use microtype if available
\IfFileExists{microtype.sty}{%
\usepackage{microtype}
\UseMicrotypeSet[protrusion]{basicmath} % disable protrusion for tt fonts
}{}
\usepackage[margin=1in]{geometry}
\usepackage{hyperref}
\hypersetup{unicode=true,
            pdfauthor={Muhammad Faris Suryadani \textbar{} Fauzan Fuadiansyah \textbar{} Muhamad Mauladi Fadillah},
            pdfborder={0 0 0},
            breaklinks=true}
\urlstyle{same}  % don't use monospace font for urls
\usepackage{natbib}
\bibliographystyle{plainnat}


\IfFileExists{parskip.sty}{%
\usepackage{parskip}
}{% else
\setlength{\parindent}{0pt}
\setlength{\parskip}{6pt plus 2pt minus 1pt}
}
\setlength{\emergencystretch}{3em}  % prevent overfull lines
\providecommand{\tightlist}{%
  \setlength{\itemsep}{0pt}\setlength{\parskip}{0pt}}
\setcounter{secnumdepth}{5}
% Redefines (sub)paragraphs to behave more like sections
\ifx\paragraph\undefined\else
\let\oldparagraph\paragraph
\renewcommand{\paragraph}[1]{\oldparagraph{#1}\mbox{}}
\fi
\ifx\subparagraph\undefined\else
\let\oldsubparagraph\subparagraph
\renewcommand{\subparagraph}[1]{\oldsubparagraph{#1}\mbox{}}
\fi

%%% Use protect on footnotes to avoid problems with footnotes in titles
\let\rmarkdownfootnote\footnote%
\def\footnote{\protect\rmarkdownfootnote}


  \title{\large\bfseries{Template Modul}}
    \author{Muhammad Faris Suryadani \textbar{} Fauzan Fuadiansyah
\textbar{} Muhamad Mauladi Fadillah}
      \date{2025-11-13}

\usepackage[indonesian]{babel}
\usepackage{xcolor}
\usepackage{color}
\usepackage{colortbl}
\usepackage[T1]{fontenc}
\usepackage{tgpagella}
\usepackage{booktabs}
\usepackage{graphicx}
\usepackage{tikz}
\usetikzlibrary{arrows}
\usepackage{forest}
\usepackage{scrextend}
\usepackage{changepage}
\usepackage{menukeys}
\usepackage{smartdiagram}
\usepackage{tcolorbox}
\usepackage{scrextend}
\usepackage[font=small,skip=6pt]{caption}
\usetikzlibrary{positioning,shapes.misc}
\usetikzlibrary{mindmap}
\usetikzlibrary{backgrounds, decorations.fractals}
\usepackage{systeme}
\usepackage{multicol,lipsum}
\usepackage{ragged2e}
%\usepackage{biblatex}
\usepackage{subcaption} % untuk subfigure modern


\usepackage{verbatimbox}
\usepackage{listings}
\usepackage{xcolor}


\definecolor{codegreen}{rgb}{0,0.6,0}
\definecolor{codegray}{rgb}{0.5,0.5,0.5}
\definecolor{codepurple}{rgb}{0.58,0,0.82}
\definecolor{backcolour}{rgb}{0.95,0.95,0.92}

\lstdefinestyle{mystyle}{
    backgroundcolor=\color{backcolour},   
    commentstyle=\color{codegreen},
    keywordstyle=\color{magenta},
    numberstyle=\tiny\color{codegray},
    stringstyle=\color{codepurple},
    basicstyle=\ttfamily\footnotesize,
    breakatwhitespace=false,         
    breaklines=true,                 
    captionpos=b,                    
    keepspaces=true,                 
    numbers=left,                    
    numbersep=5pt,                  
    showspaces=false,                
    showstringspaces=false,
    showtabs=false,                  
    tabsize=2,
    xleftmargin=1.4cm % Menambahkan margin kiri 1 cm
}

\lstset{style=mystyle}
\renewcommand{\arraystretch}{0.8} % Mengurangi jarak antar baris


\newenvironment{myindent}
{\par\leftskip1cm\relax\rightskip1cm\relax}
{\par\leftskip0cm\relax\rightskip0cm\relax}

\newcommand{\HRule}{\rule{\linewidth}{0.5mm}}
\renewcommand\fbox{\fcolorbox{red}{white}}

\hyphenation{menetap-kan dikelompok-kan}


%%% table
\usepackage{tabularx}
\usepackage{enumitem}

\setlist{nolistsep}
\definecolor{green}{HTML}{66FF66}
\definecolor{myGreen}{HTML}{009900}

\renewcommand{\familydefault}{\sfdefault}
\renewcommand{\arraystretch}{1.5}





% change section title styling
\usepackage{sectsty}
\sectionfont{\normalsize\normalfont\textbf}
\subsectionfont{\normalsize\normalfont\textbf}

% use fancyhdr style
\usepackage{fancyhdr}
\pagestyle{fancy}
\fancyhead[LO, LE]{\large\bfseries{Template Modul}}
\fancyhead[RO, RE]{(Proposal)}
\makeatletter
\renewcommand{\maketitle}{\bgroup\vspace*{-1cm}\setlength{\parindent}{0pt}
\begin{flushleft}
  \@author
  
  \@date
  
\end{flushleft}\egroup
}
\makeatother


\begin{document}
\maketitle

{
\setcounter{tocdepth}{3}
\tableofcontents
}
\listoffigures
\begin{figure}[h]
\centering
\includegraphics[width=6.5cm]{gambar/vokasi6.png}
\end{figure}

\addtocounter{section}{0}
\setcounter{page}{1}

\newpage

\section{BAB 1: PENDAHULUAN}\label{bab-1-pendahuluan}

\subsection{1.1 Latar Belakang}\label{latar-belakang}

\subsection{1.2 Rumusan Masalah}\label{rumusan-masalah}

\begin{enumerate}
\def\labelenumi{\arabic{enumi}.}
\tightlist
\item
  Bagaimana mengidentifikasi pola pembelian pelanggan e-commerce
  berdasarkan data transaksi historis?
\item
  Bagaimana menerapkan algoritma K-Means untuk melakukan segmentasi
  pelanggan berdasarkan pola pembelian mereka?
\item
  Apa karakteristik dan profil dari setiap segmen pelanggan yang
  dihasilkan dari proses clustering?
\end{enumerate}

\subsection{1.3 Tujuan Penelitian}\label{tujuan-penelitian}

\begin{enumerate}
\def\labelenumi{\arabic{enumi}.}
\tightlist
\item
  Menganalisis dan mengidentifikasi pola pembelian pelanggan e-commerce
  berdasarkan data transaksi yang tersedia.
\item
  Mengimplementasikan algoritma K-Means untuk melakukan segmentasi
  pelanggan berdasarkan pola pembelian mereka dengan menggunakan
  fitur-fitur seperti recency, frequency, dan monetary (RFM).
\item
  Menganalisis dan mendeskripsikan karakteristik setiap segmen pelanggan
  yang dihasilkan, termasuk profil perilaku pembelian dan nilai
  pelanggan.
\end{enumerate}

\section{BAB 2 : Data}\label{bab-2-data}

\subsection{2.1 Sumber Data}\label{sumber-data}

Data yang digunakan dalam penelitian ini berasal dari platform Kaggle,
dengan judul International Sale Report. Dataset ini merupakan data
publik yang berisi catatan transaksi penjualan internasional, mencakup
informasi pelanggan, produk, ukuran, harga satuan, dan nilai penjualan
secara keseluruhan. Data ini bersifat sekunder, karena diperoleh dari
sumber terbuka dan bukan hasil pengumpulan langsung oleh peneliti.
Dataset ini terdiri dari 37.432 baris data dan 10 variabel, yaitu: 1.
index -- nomor urut data. 2. DATE -- tanggal terjadinya transaksi
penjualan. 3. Months -- periode bulan dan tahun transaksi. 4. CUSTOMER
-- nama pelanggan yang melakukan pembelian. 5. Style -- kode gaya atau
model produk yang dijual. 6. SKU -- kode stok barang unik yang mewakili
kombinasi produk. 7. Size -- ukuran produk yang dibeli (misalnya L, XL,
XXL) 8. PCS -- jumlah unit barang yang terjual dalam satu transaksi. 9.
RATE -- harga satuan produk. 10. GROSS AMT -- total nilai penjualan
kotor per transaksi. Dari segi kualitas, dataset ini tergolong cukup
baik dan terstruktur, dengan kolom yang jelas serta konsisten
antarbaris. Namun, karena berasal dari sumber publik, terdapat potensi
inkonsistensi format tanggal, nilai kosong (missing value), atau
duplikasi data, sehingga perlu dilakukan tahap pembersihan data (data
cleaning) sebelum digunakan dalam analisis lebih lanjut.

\nocite{*}

\bibliography{references.bib}


\end{document}